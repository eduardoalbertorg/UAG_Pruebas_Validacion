% PLANTILLA APA7
% Creado por: Isaac Palma Medina
% Última actualización: 25/07/2021
% @COPYLEFT

% Fuentes consultadas (todos los derechos reservados):  
% Normas APA. (2019). Guía Normas APA. https://normas-apa.org/wp-content/uploads/Guia-Normas-APA-7ma-edicion.pdf
% Tecnológico de Costa Rica [Richmond]. (2020, 16 abril). LaTeX desde cero con Overleaf (1 de 3) [Vídeo]. YouTube. https://www.youtube.com/watch?v=kM1KvHVuaTY Weiss, D. (2021). 
% Formatting documents in APA style (7th Edition) with the apa7 LATEX class. https://ctan.math.washington.edu/tex-archive/macros/latex/contrib/apa7/apa7.pdf @COPYLEFT

%+-+-+-+-++-+-+-+-+-+-+-+-+-++-+-+-+-+-+-+-+-+-+-+-+-+-+-+-+-+-++-+-+-+-+-+-+-+-+-+

% Preámbulo
\documentclass[stu, 12pt, letterpaper, donotrepeattitle, floatsintext, natbib]{apa7}
\usepackage[utf8]{inputenc}
\usepackage{comment}
\usepackage{marvosym}
\usepackage{graphicx}
\usepackage{float}
\usepackage[normalem]{ulem}
\usepackage[spanish]{babel} 
\selectlanguage{spanish}
\useunder{\uline}{\ul}{}
\newcommand{\myparagraph}[1]{\paragraph{#1}\mbox{}\\}

% Portada
\thispagestyle{empty}
\title{\Large Tarea 2}
\author{Rodríguez García Eduardo Alberto} % (autores separados, consultar al docente)
\authorsaffiliations{Universidad Autónoma de Guadalajara}
% Manera oficial de colocar los autores:
%\author{Autor(a) I, Autor(a) II, Autor(a) III, Autor(a) X}
\course{Pruebas, Validación y Verificación de Software}
\professor{Maria Guadalupe Torres Godoy}
\duedate{\today}
\begin{document}
    \maketitle


    % Índices
    \pagenumbering{roman}
    % Contenido
    \renewcommand\contentsname{\largeÍndice}
    \tableofcontents
    \setcounter{tocdepth}{2}
    \clearpage
    % Figuras
    \renewcommand{\listfigurename}{\largeÍndice de fíguras}
    \listoffigures
    \clearpage
    % Tablas
    \renewcommand{\listtablename}{\largeÍndice de tablas}
    \listoftables
    \clearpage

    % Cuerpo
    \pagenumbering{arabic}

    \section{\large Comparación entre ISTQB y libro de Pressman}

    El capítulo 2 habla sobre las pruebas durante el ciclo de vida de desarrollo del software; especificando los niveles de pruebas, los tipos de pruebas, y parte del testing que se hace en la etapa de mantenimiento, mientras que el capítulo 3 habla sobre las pruebas estáticas y el proceso que llevan.

    El capítulo 15 del libro de Pressman habla sobre las pruebas estáticas, cómo se lleva el proceso, hace referencia a los tipos de revisiones y a métricas.

    Ambas son lecturas que se complementan muy bien entre ellas, ya que ambas describen los diferentes tipos de pruebas estáticas, las ventajas, y cómo se llevan a cabo, sin embargo, cabe resaltar que el libro de Pressman entra en mayor detalle hacia los procesos, los tipos y las métricas.

    \section{Dudas respecto a la lectura}

    En el libro de Pressman mencionan métricas y análisis de estas métricas donde por ejemplo, se da a conocer la densidad del error. Mis preguntas son: 
    \begin{enumerate}
        \item ¿Esto sirve para todos los equipos?
        \item ¿Esto se aplica incluso en equipos nuevos?
        \item ¿Qué pasa si el equipo1 cambia constantemente de miembros?
        \item ¿Qué pasa si solo cambia 1 o pocos miembros del equipo1, se mantiene igual las métricas?
    \end{enumerate}
    

\end{document}